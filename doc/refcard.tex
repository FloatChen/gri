% scp refcard.ps dankelley@gri.sourceforge.net:/home/groups/g/gr/gri/htdocs

\def\griversion{2.8}
\def\date{2001 Apr 11}

% Reference Card for Gri
%**start of header
\hbadness=10000
\vbadness=10000
\newcount\columnsperpage

% This file can be printed with 1, 2, or 3 columns per page (see below).
% Specify how many you want here.  Nothing else needs to be changed.

\columnsperpage=3

% This file is intended to be processed by plain TeX (TeX82).
%
% The final reference card has six columns, three on each side.
% This file can be used to produce it in any of three ways:
% 1 column per page
%    produces six separate pages, each of which needs to be reduced to 80%.
%    This gives the best resolution.
% 2 columns per page
%    produces three already-reduced pages.
%    You will still need to cut and paste.
% 3 columns per page
%    produces two pages which must be printed sideways to make a
%    ready-to-use 8.5 x 11 inch reference card.
%    For this you need a dvi device driver that can print sideways, e.g.,
%      lw -land griref.dvi
% Which mode to use is controlled by setting \columnsperpage above.
%
% Author:
%  Dan Kelley (copied from a Unix reference card written by Steven Matheson)
%  Internet: matehson@open.dal.ca
%
% Note by Peter Galbraith, July 2001:  It's pretty obvious to me that Steve
% based his reference card on emacs' refcard written by Stephen Gildea.
%  Therefore, this file is a derived product and is licensed under the GPL.

\def\version{\date}

\def\shortnotice{\vskip 1ex plus 2 fill
  \centerline{\small Gri version \versionnumber}}

\def\notice{
\vskip 1ex plus 2 fill\begingroup\small
\centerline{(c) 2001, Dan E. Kelley}
See also {\it cmdrefcard}, a companion reference card.
\endgroup}

% make \bye not \outer so that the \def\bye in the \else clause below
% can be scanned without complaint.
\def\bye{\par\vfill\supereject\end}

\newdimen\intercolumnskip
\newbox\columna
\newbox\columnb

\def\ncolumns{\the\columnsperpage}

\message{[\ncolumns\space 
V  column\if 1\ncolumns\else s\fi\space per page]}

\def\scaledmag#1{ scaled \magstep #1}

% This multi-way format was designed by Stephen Gildea
% October 1986.
\if 1\ncolumns
  \hsize 4in
  \vsize 10in
  \voffset -.7in
  \font\titlefont=\fontname\tenbf \scaledmag3
  \font\headingfont=\fontname\tenbf \scaledmag2
  \font\smallfont=\fontname\sevenrm
  \font\smallsy=\fontname\sevensy

  \footline{\hss\folio}
  \def\makefootline{\baselineskip10pt\hsize6.5in\line{\the\footline}}
\else
  \hsize 3.2in
  \vsize 7.95in
  \hoffset -.75in
  \voffset -.745in
  \font\titlefont=cmbx10 \scaledmag2
  \font\headingfont=cmbx10 \scaledmag1
  \font\subheadingfont=cmbx10 \scaledmag0
  \font\smallfont=cmr6
  \font\smallsy=cmsy6
  \font\eightrm=cmr8
  \font\eightbf=cmbx8
  \font\eightit=cmti8
  \font\eighttt=cmtt8
  \font\eightsy=cmsy8
  \textfont0=\eightrm
  \textfont2=\eightsy
  \def\rm{\eightrm}
  \def\bf{\eightbf}
  \def\it{\eightit}
  \def\tt{\eighttt}
  \normalbaselineskip=.8\normalbaselineskip
  \normallineskip=.8\normallineskip
  \normallineskiplimit=.8\normallineskiplimit
  \normalbaselines\rm           %make definitions take effect

  \if 2\ncolumns
    \let\maxcolumn=b
    \footline{\hss\rm\folio\hss}
    \def\makefootline{\vskip 2in \hsize=6.86in\line{\the\footline}}
  \else \if 3\ncolumns
    \let\maxcolumn=c
    \nopagenumbers
  \else
    \errhelp{You must set \columnsperpage equal to 1, 2, or 3.}
    \errmessage{Illegal number of columns per page}
  \fi\fi

  \intercolumnskip=.46in
  \def\abc{a}
  \output={%
      % This next line is useful when designing the layout.
      %\immediate\write16{Column \folio\abc\space starts with \firstmark}
      \if \maxcolumn\abc \multicolumnformat \global\def\abc{a}
      \else\if a\abc
        \global\setbox\columna\columnbox \global\def\abc{b}
        %% in case we never use \columnb (two-column mode)
        \global\setbox\columnb\hbox to -\intercolumnskip{}
      \else
        \global\setbox\columnb\columnbox \global\def\abc{c}\fi\fi}
  \def\multicolumnformat{\shipout\vbox{\makeheadline
      \hbox{\box\columna\hskip\intercolumnskip
        \box\columnb\hskip\intercolumnskip\columnbox}
      \makefootline}\advancepageno}
  \def\columnbox{\leftline{\pagebody}}

  \def\bye{\par\vfill\supereject
    \if a\abc \else\null\vfill\eject\fi
    \if a\abc \else\null\vfill\eject\fi
    \end}  
\fi

% we won't be using math mode much, so redefine some of the characters
% we might want to talk about
\catcode`\^=12
\catcode`\_=12

\chardef\\=`\\
\chardef\{=`\{
\chardef\}=`\}

\hyphenation{mini-buf-fer}

\parindent 0pt
\parskip 1ex plus .5ex minus .5ex

\def\small{\smallfont\textfont2=\smallsy\baselineskip=.8\baselineskip}

\def\n{\hfil\break}

\outer\def\newcolumn{\vfill\eject}

\outer\def\title#1{{\titlefont\centerline{#1}}\vskip 1ex plus .5ex}

\outer\def\section#1{\par\filbreak
  \vskip 3ex plus 2ex minus 2ex {\headingfont #1}\mark{#1}%
  \vskip 2ex plus 1ex minus 1.5ex}

\outer\def\subsection#1{\par\filbreak
  \vskip 3ex plus 2ex minus 2ex {\subheadingfont #1}\mark{#1}%
  \vskip 2ex plus 1ex minus 1.5ex}

\newdimen\keyindent

\def\beginindentedkeys{\keyindent=1em}
\def\endindentedkeys{\keyindent=0em}
\endindentedkeys

\def\paralign{\vskip\parskip\halign}

\def\<#1>{$\langle${\rm #1}$\rangle$}

\def\kbd#1{{\tt#1}\null}        %\null so not an abbrev even if period follows

\def\beginexample{\leavevmode\begingroup
  \obeylines\obeyspaces\parskip0pt\tt}
{\obeyspaces\global\let =\ }
\def\endexample{\endgroup}

\def\key#1#2{\leavevmode\hbox to \hsize{\vtop
  {\hsize=.65\hsize\rightskip=1em
  \hskip\keyindent\relax#1}\kbd{#2}\hfil}}

\newbox\metaxbox
\setbox\metaxbox\hbox{\kbd{M-x }}
\newdimen\metaxwidth
\metaxwidth=\wd\metaxbox

\def\metax#1#2{\leavevmode\hbox to \hsize{\hbox to .75\hsize
  {\hskip\keyindent\relax#1\hfil}%
  \hskip -\metaxwidth minus 1fil
  \kbd{#2}\hfil}}

\def\threecol#1#2#3{\hskip\keyindent\relax#1\hfil&\kbd{#2}\quad
  &\kbd{#3}\quad\cr}

%**end of header


\tolerance=10000
\title{Gri \griversion$\,$Reference Card}



\section{1\quad What Gri Is}
Gri is a language for drawing scientific diagrams such as x-y graphs,
contours, vector fields, and images.  The Gri language is extensible,
well-tested and fully documented.  The output is in the PostScript
page description language.

\section{2\quad How to Run Gri}
Normally Gri is run non-interactively.  At the system prompt, type
{\tt gri foo.gri} to run Gri on the file {\tt foo.gri}, creating a
PostScript file called {\tt foo.ps}.  Several command-line options
exist; type {\tt gri -help} to see them, or consult the manual.

Occasionally you might want to run Gri interactively.  To do this,
type {\tt gri} at the system prompt, and then type Gri commands at the
Gri prompt, using {\tt quit} to get out of Gri.


\section{3\quad Overview of Gri Language}
\subsection{3.1\quad Syntax}
Commands normally appear one per line, although ending a line with
back-slash causes Gri to scan the next line also.  Comments may be
inserted at the end of non-continued lines by preceeding the comment
by a hash-code ({\tt \#}).

Gri allows the usual suite of programming structures, such as loops
and if-statements; additionally, new commands may be added to Gri
easily (see section 3.7).

\subsection{3.2\quad Built-in Commands}

Here are the first words of the built-in Gri commands:
\beginexample
cd      close         convert     create    debug
delete  differentiate draw        expecting filter
flip    get           help        if        ignore
input   insert        interpolate list      ls
mask    move          new         open      pwd
query   quit          read        regress   reorder
rescale resize        return      rewind    set
show    skip          smooth      sprintf   superuser
system  write
\endexample
To get more information on a given command, e.g. the {\tt open}
command, type {\tt help open} in an interactive Gri session, or {\tt
C-H i gri commands open} in an {\tt emacs} editing session, or {\tt
info commands open} at the system level.


\subsection{3.3\quad Mathematics}
Wherever Gri expects to see a number in a command, one may substitute
a mathematical expression written in reverse polish notation (RPN)
notation.  RPN expressions are enclosed in braces and preceeded by the
word {\tt rpn}, e.g.
\beginexample
set x size $\lbrace$ rpn 5 2.54 * $\rbrace$ \# Make width be 5 inches
x += $\lbrace$ rpn 1 2 /$\rbrace$           \# Add 0.5 to x values
y -= $\lbrace$ rpn y mean $\rbrace$         \# De-mean y column
x -= $\lbrace$ rpn x 0 @ $\rbrace$          \# Subtract first value
\endexample



\subsection{3.4\quad Variables (for Storing Numbers)}
User-defined variables have names that begin and end with periods
(like {\tt .offset.}); variables defined by gri (which you may alter
if you wish) have names that begin and end with two periods (like {\tt
..xsize..}).  To list the variables use {\tt show variables}.  Each of
the following commands accomplishes the same thing, making the plot 2
cm wider.  (Gri uses {\tt ..xsize..} to store the width of the plot.)
\beginexample
..xsize.. = $\lbrace$ rpn ..xsize.. 2.0 + $\rbrace$
..xsize.. += 2
set x size $\lbrace$ rpn ..xsize.. 2.0 + $\rbrace$
set x size bigger 2
\endexample



\subsection{3.5\quad Synonyms (for Storing Strings)}
Synonyms have names which begin with backslash (like {\tt
$\backslash$name}).  To list the synonyms use {\tt show synonyms}.
Synonyms can be embedded within strings or used raw, e.g.
\beginexample
\\dir = "mydir"
query \\filename "What's the data file?" ("file.dat")
open \\mydir/\\filename
read columns x y
draw curve
draw title "Data in \\filename in dir \\mydir"
\endexample



\subsection{3.6\quad Strings and Math Symbols}
Strings are enclosed in double quotes.  As in \TeX, superscripts and
subscripts are enclosed in dollar signs.  Subscripts are preceeded by
underscore, superscripts by carat.  Superscripts or subscripts
consisting of more than one character are enclosed in braces.

Gri handles Greek letters and mathematical symbols as \TeX\ does: they
are enclosed in dollar signs and have backslash as the first
character.  Most Greek letters are available, along with several
mathematical symbols, but complicated La\TeX\ macros (like {\tt
$\backslash$frac$\lbrace\rbrace\lbrace\rbrace$}) are not available.
Examples:
\beginexample
set x name "x/x\$_0\$"
draw title "y\$_$\lbrace$dim$\rbrace$\$ as fcn of \$\\alpha\$"
\endexample



\subsection{3.7\quad Extending Gri}
You can create new Gri commands in your commandfile or in your {\tt
$\sim$/.grirc} file.  Commands in {\tt $\sim$/.grirc} can be used
anywhere -- they are your personal extensions to gri.  New commands
defined in your commandfile exist only within that file.  The example
below defines a new command which is invoked by {\tt Landscape Big}.
The command name (which should begin with upper case letters, to avoid
clashing with future built-in commands in gri) is enclosed in angled
single-quotes.  Optional help lines follow.  The body of the command
starts after a line with an opening brace, and ends before a line with
a closing brace.
\beginexample
`Landscape Big'
Plot in landscape mode, big size
$\lbrace$
    set page landscape
    set x margin 2
    set x size 25
    set y margin 2
    set y size 15
$\rbrace$
open file.dat
read columns x y
close
Landscape Big
draw curve
quit
\endexample

\section{4\quad Editing Gri in GNU Emacs}
A {\tt gri-mode} is available for editing Gri commandfiles in {\tt emacs}.
It provides completion of Gri commands, a quick interface to the gri manual
about the command being edited, useful pulldown menus, code
fontification and the usual indentation, comment placement, etc.
Consult the Gri manual for a full description.  To use {\tt gri-mode}, put
lines like the following in your {\tt $\sim$/.emacs} file.
\beginexample
;;; Gri mode
(autoload 'gri-mode "gri-mode" "Enter Gri-mode." t)
(setq auto-mode-alist 
    (cons '("\\.gri\$" . gri-mode) auto-mode-alist))
\endexample

\section{5\quad Documentation and User Group}
An {\tt info} manual is available at the system level and inside
Emacs.  A PostScript manual is also online, along with a cookbook of
Gri examples.  Since Unix has no standard place to store PostScript
manuals, you must consult your system manager to find these files.  A
World Wide Web manual is available at {\it
  http://gri.sourceforge.net}.
There are Gri mailing lists and discussion groups at this site as
well.


\section{6.\quad Example -- Linegraph}
Suppose the file {\tt example1.dat} contains data in two columns
separated by white space.  The following shows how to plot data with
lines connecting the points.  To get symbols without lines, substitute
{\tt draw symbols} for {\tt draw curve}; to get both symbols and
lines, use both {\tt draw} commands.  If you have several curves which
cross, use {\tt draw curve overlying}, which whites out a border below
each curve, yielding a visual cue that lets the eye trace the
individual curves easily.
\beginexample
\# Example1.gri -- linegraph using data in a file
open example1.dat      \# Open the data file
read columns x y       \# Read (x,y)
draw curve             \# Draw curve stored in (x,y)
\endexample

You'll notice that there was no need to ask that axes be drawn.  They
will be automatically determined (based on the data that were read in)
and drawn just after the {\tt draw curve} command.  (Gri likes to draw
axes, and you've got to ask it not to do so, if you don't want them.)
Also, note that Gri was not instructed to close the datafile.  This is
done automatically at termination.  One could insert a {\tt close}
command after the {\tt read} command, if desired; this is helpful when
you wish to work with many data files sequentially.

The axes are labelled {\tt x} and {\tt y}.  To change that, and to add
a title, do as follows:
\beginexample
open example1.dat
read columns x y
set x name "Time, s"
set y name "Distance, m"
draw curve
draw title "Trajectory of fluid motion"
\endexample

To get a thicker curve, say 2 points wide, you could do
\beginexample
open example1.dat
read columns x y
set line width 2
draw curve
\endexample

To get a dashed line,
\beginexample
open example1.dat
read columns x y
set dash
draw curve
\endexample

To get a red line,
\beginexample
open example1.dat
read columns x y
draw axes
set color red
draw curve
\endexample

Note in the above that the axes were drawn before the color was set to
red, so they will come out black.  Otherwise both the curve and the
axes would be red.


\section{7.\quad Example -- Contour Graph}
Gri can plot contour graphs of either gridded or ungridded data.
Several methods are provided for gridding data.  The following example
shows how to grid randomly distributed (x,y,z) data and plot contours.
\beginexample
\# Example 5 - Contouring ungridded data, from figure
\# 5 of Koch et al., 1983, J. Climate Appl. Met.,
\# volume 22, pages 1487-1503.
open example5.dat
read columns x y z
close
set x size 12
set x axis 0 12 2
set y size 10
set y axis 0 10 2
draw axes
set line width symbol 0.2
set symbol size 0.2
draw symbol bullet
set font size 8
draw values
set x grid 0 12 0.25
set y grid 0 10 0.25

convert columns to grid 

\# Uncomment next line to smooth the grid:
\#smooth grid data

set font size 10
draw contour 0 40 2
set font size 12
draw title "Data from Fig 5 Koch et al., 1983"
quit
\endexample

Note that a {\tt quit} command has been included, although it is not
required, since Gri quits when it reaches the end of the commandfile
anyway.




\section{8.\quad Example -- Image Graph} Gri can draw images in BW and color.
The following example shows how to plot a satellite image.


\beginexample
\# Example 6 -- Plot IR image of Gulf of Maine
\# define characteristics of norda images
\\0val         = "5"             \# 0 in image
\\255val       = "30.5"          \# 255 in image
.r.           = 128             \# rows
.c.           = 128             \# cols
.pixel_width. = 2
.km.          = $\lbrace$rpn .c. .pixel_width. *$\rbrace$

\# get filenames
query \\filename "Image file?" ("example6image.dat")
query \\maskname "Mask  file?" ("example6mask.dat")

\# get data
open \\filename binary
set image range \\0val \\255val
read image .r. .c. box 0 0 .km. .km.
close
open \\maskname binary
read image mask .r. .c.
close

\# find out what grayscale method to use
query \\histo "Flatten histogram?"       ("no" "yes")
query \\Tw    "T/deg for white on page?" ("10")
query \\Tb    "T/deg for black on page?" ("15")

\# set up scales. 
set x size 12.8
set y size 12.8
set x name "km"
set y name "km"
set x axis 0 .km. 32
set y axis 0 .km. 32

\# plot image, grayscale, and histogram
if $\lbrace$rpn \\histo "yes" ==$\rbrace$
    set image grayscale using histogram \\
        black \\Tb white \\Tw
else
    set image grayscale black \\Tb white \\Tw
end if
draw image
draw image palette left \\Tw right \\Tb
draw image histogram
if $\lbrace$rpn \\histo "yes" == $\rbrace$
    draw title "Grayscale histogram enhanced"
else
    draw title "Grayscale linear \\Tw to \\Tb"
end if
\endexample


\notice

\bye
