% $Id: cmdrefcard.tex,v 1.1 2000/05/11 15:07:21 dankelley Exp $

\def\griversion{2.0.x}
\def\date{1999 Jan 28}

% Reference Card for Gri
%**start of header
\hbadness=10000
\vbadness=10000
\newcount\columnsperpage

% This file can be printed with 1, 2, or 3 columns per page (see below).
% Specify how many you want here.  Nothing else needs to be changed.

\columnsperpage=3

% This file is intended to be processed by plain TeX (TeX82).
%
% The final reference card has six columns, three on each side.
% This file can be used to produce it in any of three ways:
% 1 column per page
%    produces six separate pages, each of which needs to be reduced to 80%.
%    This gives the best resolution.
% 2 columns per page
%    produces three already-reduced pages.
%    You will still need to cut and paste.
% 3 columns per page
%    produces two pages which must be printed sideways to make a
%    ready-to-use 8.5 x 11 inch reference card.
%    For this you need a dvi device driver that can print sideways, e.g.,
%      lw -land griref.dvi
% Which mode to use is controlled by setting \columnsperpage above.
%
% Author:
%  Dan Kelley (copied from a Unix reference card written by Steven Matheson)
%  Internet: matehson@open.dal.ca
%

\def\shortnotice{\vskip 1ex plus 2 fill
  \centerline{\small Gri version \versionnumber}}

\def\notice{
\vskip 1ex plus 2 fill\begingroup\small
\centerline{(c) 1999, Dan E. Kelley}

This PostScript document is available on the World Wide Web at {\it
ftp://ftp.phys.ocean.dal.ca/pub/receive/kelley/gri/cmdrefcard.ps}.
See also {\it refcard.ps}, which is a companion reference card, as
well as the full manual and source/binary code in this directory.
Please inform me ({\tt Dan.Kelley@Dal.Ca}) of errors or deficiencies
in this document or in Gri, so I can make improvements.
\endgroup}

% make \bye not \outer so that the \def\bye in the \else clause below
% can be scanned without complaint.
\def\bye{\par\vfill\supereject\end}

\newdimen\intercolumnskip
\newbox\columna
\newbox\columnb

\def\ncolumns{\the\columnsperpage}

\message{[\ncolumns\space 
V  column\if 1\ncolumns\else s\fi\space per page]}

\def\scaledmag#1{ scaled \magstep #1}

% This multi-way format was designed by Stephen Gildea
% October 1986.
\if 1\ncolumns
  \hsize 4in
  \vsize 10in
  \voffset -.7in
  \font\titlefont=\fontname\tenbf \scaledmag3
  \font\headingfont=\fontname\tenbf \scaledmag2
  \font\smallfont=\fontname\sevenrm
  \font\smallsy=\fontname\sevensy

  \footline{\hss\folio}
  \def\makefootline{\baselineskip10pt\hsize6.5in\line{\the\footline}}
\else
  \hsize 3.2in
  \vsize 7.95in
  \hoffset -.75in
  \voffset -.745in
  \font\titlefont=cmbx10 \scaledmag2
  \font\headingfont=cmbx10 \scaledmag1
  \font\subheadingfont=cmbx10 \scaledmag0
  \font\smallfont=cmr6
  \font\smallsy=cmsy6
  \font\eightrm=cmr8
  \font\eightbf=cmbx8
  \font\eightit=cmti8
  \font\eighttt=cmtt8
  \font\eightsy=cmsy8
  \textfont0=\eightrm
  \textfont2=\eightsy
  \def\rm{\eightrm}
  \def\bf{\eightbf}
  \def\it{\eightit}
  \def\tt{\eighttt}
  \normalbaselineskip=.8\normalbaselineskip
  \normallineskip=.8\normallineskip
  \normallineskiplimit=.8\normallineskiplimit
  \normalbaselines\rm           %make definitions take effect

  \if 2\ncolumns
    \let\maxcolumn=b
    \footline{\hss\rm\folio\hss}
    \def\makefootline{\vskip 2in \hsize=6.86in\line{\the\footline}}
  \else \if 3\ncolumns
    \let\maxcolumn=c
    \nopagenumbers
  \else
    \errhelp{You must set \columnsperpage equal to 1, 2, or 3.}
    \errmessage{Illegal number of columns per page}
  \fi\fi

  \intercolumnskip=.46in
  \def\abc{a}
  \output={%
      % This next line is useful when designing the layout.
      %\immediate\write16{Column \folio\abc\space starts with \firstmark}
      \if \maxcolumn\abc \multicolumnformat \global\def\abc{a}
      \else\if a\abc
        \global\setbox\columna\columnbox \global\def\abc{b}
        %% in case we never use \columnb (two-column mode)
        \global\setbox\columnb\hbox to -\intercolumnskip{}
      \else
        \global\setbox\columnb\columnbox \global\def\abc{c}\fi\fi}
  \def\multicolumnformat{\shipout\vbox{\makeheadline
      \hbox{\box\columna\hskip\intercolumnskip
        \box\columnb\hskip\intercolumnskip\columnbox}
      \makefootline}\advancepageno}
  \def\columnbox{\leftline{\pagebody}}

  \def\bye{\par\vfill\supereject
    \if a\abc \else\null\vfill\eject\fi
    \if a\abc \else\null\vfill\eject\fi
    \end}  
\fi

% we won't be using math mode much, so redefine some of the characters
% we might want to talk about
\catcode`\^=12
\catcode`\_=12

\chardef\\=`\\
\chardef\{=`\{
\chardef\}=`\}

\hyphenation{mini-buf-fer}

\parindent 0pt
\parskip 1ex plus .5ex minus .5ex

\def\small{\smallfont\textfont2=\smallsy\baselineskip=.8\baselineskip}

\def\n{\hfil\break}

\outer\def\newcolumn{\vfill\eject}

\outer\def\title#1{{\titlefont\centerline{#1}}\vskip 1ex plus .5ex}

\outer\def\section#1{\par\filbreak
  \vskip 3ex plus 2ex minus 2ex {\headingfont #1}\mark{#1}%
  \vskip 2ex plus 1ex minus 1.5ex}

\outer\def\subsection#1{\par\filbreak
  \vskip 3ex plus 2ex minus 2ex {\subheadingfont #1}\mark{#1}%
  \vskip 2ex plus 1ex minus 1.5ex}

\newdimen\keyindent

\def\beginindentedkeys{\keyindent=1em}
\def\endindentedkeys{\keyindent=0em}
\endindentedkeys

\def\paralign{\vskip\parskip\halign}

\def\<#1>{$\langle${\rm #1}$\rangle$}

\def\kbd#1{{\tt#1}\null}        %\null so not an abbrev even if period follows

\def\beginexample{\par\leavevmode\begingroup
%17feb93  \obeylines\obeyspaces\parskip0pt\tt}
  \obeylines\obeyspaces\parskip0pt\vskip-12pt\tt}
{\obeyspaces\global\let =\ }
\def\endexample{\endgroup}

\def\key#1#2{\leavevmode\hbox to \hsize{\vtop
  {\hsize=.65\hsize\rightskip=1em
  \hskip\keyindent\relax#1}\kbd{#2}\hfil}}

\newbox\metaxbox
\setbox\metaxbox\hbox{\kbd{M-x }}
\newdimen\metaxwidth
\metaxwidth=\wd\metaxbox

\def\metax#1#2{\leavevmode\hbox to \hsize{\hbox to .75\hsize
  {\hskip\keyindent\relax#1\hfil}%
  \hskip -\metaxwidth minus 1fil
  \kbd{#2}\hfil}}

\def\threecol#1#2#3{\hskip\keyindent\relax#1\hfil&\kbd{#2}\quad
  &\kbd{#3}\quad\cr}

%**end of header

\tolerance=10000
\title{Overview of Gri \griversion$\,$Commands}
\raggedright

\section{1\quad Introduction}
This reference card describes the commands in version \griversion\ of
the Gri plotting language.  See also the companion ``Gri Reference
Card'' and the online manuals.

\section{2\quad Control Statements}

\subsection{2.1\quad If Statements}
The {\tt if} statement has ancillary {\tt else if} and {\tt else}
statements, and is ended by the {\tt end if} statement, e.g.
\beginexample
if $\lbrace$rpn .x. 10 >$\rbrace$
    show "The variable .x. is less than 10"
else if $\lbrace$rpn .x. 20 >$\rbrace$
    show "The variable .x. is between 10 and 20"
else
   show "The variable .x. is greater than 20"
end if
\endexample


\subsection{2.2\quad Loops}
{\tt while} loops are provided.  The statements between {\tt while}
and {\tt end while} are repeated until the RPN expression on the {\tt
while} commandline is false.

Here is an infinite loop ended -- by a {\tt break} statement -- when
the file contents are exhausted:
\beginexample
while 1
    read .x. .y.
    if ..eof..
        break
    end if
    show ".x. is " .x.
end while
\endexample

Here is a loop that will print the numbers 0, 1, ..., 9.
\beginexample
.i. = 0
while $\lbrace$rpn .i. 10 >$\rbrace$
    show .i.
    .i. += 1
end while
\endexample



\section{3\quad List of Gri Commands}

What follows is a complete list of built-in Gri commands.  For more
help on a given command, see the full manual, or use the Gri online
help facitilty (e.g., type {\tt gri} to launch Gri, then type {\tt
help}; exit by typing

The notation is as follows.  

$\bullet$ Items written within square brackets are optional.

$\bullet$ Items written within dots are either raw numbers, RPN
expressions, or variable names.

$\bullet$ Items preceeded by backslashes are any given string.

$\bullet$ Items separated by vertical bars are alternatives.

$\bullet$ Curly brackets group words that must appear together.

Thus, for example, the syntax
\beginexample
    set dash [.n.$\mid$$\lbrace$.dash. .blank.$\rbrace$$\mid$off]
\endexample
means that {\tt set dash} is a possible Gri command (meaning use the
default dash style).  Several forms of optional items may be present
also.  For example, {\tt set dash 2} is legal; it means use the dash
style numbered 2.  Gri will check any single number presented in this
place on this command against the list of acceptable {\tt .n.} values.
If two numbers are present, Gri interprets the first as the length of
dashes and the second as the length of blanks; notice the braces,
indicating that these two parameters must appear together.  Finally,
the keyword {\tt off} is allowed (it means go back to a solid line).


\bigskip
Here are the commands:
\smallskip

\parindent -1ex

\beginexample

% INSERT stuff from cmdrefcard.pl schellscript here (to line 'end of INSERT')

cd [$\backslash$pathname]

close [$\backslash$filename]

convert columns to grid [neighbor $\mid$ $\lbrace$objective$\mid$boxcar .xr. .yr. [.n. .e.]$\rbrace$ $\mid$ $\lbrace$barnes [.xr. .yr. .gamma. .iter.]$\rbrace$]

convert columns to spline [.gamma.] [.xmin. .xmax. .xinc.]

convert columns from polar to rectangular

convert columns from rectangular to polar

convert grid to columns

convert grid to image [size .width. .height.] [box .ll\_x. .ll\_y. .ur\_x. .ur\_y.]

convert image to grid

create columns from function

create image grayscale banded .band.

create image greyscale banded .band.

debug [.n.] $\mid$ [clipped values in draw commands] $\mid$ off

delete $\lbrace$.variable. $\mid$ $\backslash$synonym [...]$\rbrace$ $\mid$ columns [$\lbrace$randomly .fraction.$\rbrace$$\mid$$\lbrace$where missing$\rbrace$] $\mid$ grid $\mid$ $\lbrace$[x$\mid$y] scale$\rbrace$

differentiate $\lbrace$x$\mid$y wrt index$\mid$y$\mid$x$\rbrace$ $\mid$ $\lbrace$grid wrt x$\mid$y$\rbrace$

draw arrow from .x0. .y0. to .x1. .y1. [cm]

draw arrows

draw axes if needed

draw axes [.style.$\mid$frame$\mid$none]

draw border box [.ll\_x. .ll\_y. .ur\_x. .ur\_y. .width\_cm. .brightness.]

draw box filled .ll\_x. .ll\_y. .ur\_x. .ur\_y. [cm]

draw box .ll\_x. .ll\_y. .ur\_x. .ur\_y. [cm]

draw circle with radius .r\_cm. at .x\_cm. .y\_cm.

draw contour [$\lbrace$.value. [unlabelled$\mid$$\lbrace$labelled "$\backslash$label"$\rbrace$]$\rbrace$ $\mid$ $\lbrace$.min. .max. .inc. [.inc\_unlabelled.] [unlabelled]$\rbrace$]

draw curve overlying

draw curve filled [to $\lbrace$.y. y$\rbrace$ $\mid$ $\lbrace$.x. x$\rbrace$]

draw curve

draw essay "text"$\mid$reset

draw gri logo [size .fontsize. at .xcm. .ycm.]

draw gri logo2 size .fontsize. at .xcm. .ycm. $\backslash$topcolor $\backslash$bottomcolor

draw grid

draw image palette [axisleft$\mid$axisright$\mid$axistop$\mid$axisbottom] [left .left. right .right. [increment .inc.]] [box .ll\_x\_cm. .ll\_y\_cm. .ur\_x\_cm. .ur\_y\_cm.]

draw image grayscale [left .left. right .right. [increment .inc.]] [box .ll\_x\_cm. .ll\_y\_cm. .ur\_x\_cm. .ur\_y\_cm.]

draw image histogram [box .ll\_x\_cm. .ll\_y\_cm. .ur\_x\_cm. .ur\_y\_cm.]

draw image

draw isopycnal .density. [unlabelled]

draw isopycnal1 .density. [unlabelled]

draw isospice .spice. [unlabelled]

draw label boxed "$\backslash$string" at .ll\_x. .ll\_y. [cm]

draw label whiteunder "$\backslash$string" at .ll\_x. .ll\_y. [cm]

draw label for last curve "label"

draw label "$\backslash$string" [centered$\mid$rightjustified] at .x. .y. [cm] [rotated .deg.]

draw line from .x0. .y0. to .x1. .y1. [cm]

draw line legend "label" at .x. .y. [cm] [length .cm.]

draw lines $\lbrace$vertically .left. .right. .inc.$\rbrace$ $\mid$ $\lbrace$horizontally .bottom. .top. .inc.$\rbrace$

draw patches .width. .height. [cm]

draw polygon [filled] .x0. .y0. .x1. .y1. .x2. .y2. [...]

draw regression line

draw symbol legend $\backslash$symbol\_name "label" at .x. .y. [cm]

draw symbol .code.$\mid$$\backslash$name at .x. .y. [cm]

draw symbol [[.code.$\mid$$\backslash$name] [color [hue z$\mid$.h.] [brightness z$\mid$.b.] [saturation z$\mid$.s.]]]

draw time stamp [fontsize .points. [at .x\_cm. .y\_cm. cm [with angle .deg.]]]

draw title "$\backslash$string"

draw values [.dx. .dy.] [$\backslash$format] [separation .xcm. .ycm.]

draw x axis [at bottom$\mid$top$\mid$$\lbrace$.y. [cm]$\rbrace$ [lower$\mid$upper]]

draw x box plot at .y. [size .cm.]

draw y axis [at left$\mid$right$\mid$$\lbrace$.x. cm$\rbrace$ [left$\mid$right]]

draw y box plot at .x. [size .cm]

draw zero line [horizontally$\mid$vertically]

expecting version .n.

filter column x$\mid$y$\mid$z$\mid$u$\mid$v$\mid$r$\mid$theta recursively .a0. .a1. ... .b0. .b1. ...

filter grid rows$\mid$columns recursively .a0. .a1. ... .b0. .b1. ...

filter image highpass$\mid$lowpass

flip grid$\mid$image x$\mid$y

get env $\backslash$result $\backslash$environment\_variable

heal columns$\mid$$\lbrace$grid along x$\mid$y$\rbrace$

help [*$\mid$command\_name$\mid$$\lbrace$- topic$\rbrace$]

if $\lbrace$[!] .flag.$\rbrace$$\mid$$\backslash$flag$\mid$$\lbrace$$\lbrace$"string1" == "string2"$\rbrace$$\rbrace$

ignore last .n.

input $\backslash$ps\_filename [.xcm. .ycm. [.xmag. .ymag. [.rot\_deg.]]]

insert $\backslash$filename

interpolate x$\mid$y grid to ...

list $\backslash$command-syntax

ls [$\backslash$file\_specification]

mask image [to $\lbrace$uservalue .u.$\rbrace$$\mid$$\lbrace$imagevalue .i.$\rbrace$]

new page

new .variable\_name.$\mid$$\backslash$synonym\_name [.variable\_name.$\mid$$\backslash$synonym\_name [...]]

open $\lbrace$$\backslash$filename$\rbrace$$\mid$$\lbrace$"system command$\mid$"$\rbrace$ $\lbrace$[binary [uchar$\mid$int$\mid$float$\mid$double]]$\rbrace$$\mid$$\lbrace$netCDF$\rbrace$

postscript $\backslash$string

pwd

query $\backslash$synonym$\mid$.variable ["$\backslash$prompt" [("$\backslash$default"$\mid$.default)]]

quit [.exit\_status.]

read colornames from RGB $\backslash$filename

read columns ...

read grid $\lbrace$x [.rows.$\mid$$\lbrace$="name"$\rbrace$]$\rbrace$$\mid$$\lbrace$y [.cols.]$\lbrace$="name"$\rbrace$$\rbrace$$\mid$$\lbrace$data $\lbrace$[spacers] [.rows. .cols.] [spacers] [bycolumns]$\rbrace$$\mid$$\lbrace$="name"$\rbrace$$\rbrace$

read image colorscale [rgb$\mid$hsb]

read image grayscale

read image greyscale

read image mask rasterfile

read image mask .rows. .cols.

read image pgm [box .ll\_x. .ll\_y. .ur\_x. .ur\_y.]

read image rasterfile [box .ll\_x. .ll\_y. .ur\_x. .ur\_y.]

read image .rows. .cols. [box .ll\_x. .ll\_y. .ur\_x. .ur\_y.] [bycolumns]

read from $\backslash$filename

read line $\backslash$synonym

read [* [*...]] $\backslash$synonym$\mid$$\lbrace$.variable. [.variable. ...]$\rbrace$

regress $\lbrace$y vs x [linear]$\rbrace$$\mid$$\lbrace$x vs y [linear]$\rbrace$

reorder columns randomly$\mid$$\lbrace$ascending in x$\mid$y$\mid$z$\rbrace$$\mid$$\lbrace$descending in x$\mid$y$\mid$z$\rbrace$

rpnfunction $\backslash$name "action"

rescale

resize x for maps

resize y for maps

return

rewind

set axes style .style. $\mid$ $\lbrace$offset [.dist\_cm.]$\rbrace$ $\mid$ rectangular $\mid$ polar $\mid$ none $\mid$ default

set arrow size .size.$\mid$$\lbrace$as .num. percent of length$\rbrace$$\mid$default

set arrow type .which.

set beep on$\mid$off

set clip [postscript] $\lbrace$on [.xleft. .xright. .ybottom. .ytop.]$\rbrace$$\mid$off

set color $\backslash$name$\mid$$\lbrace$rgb .red. .green. .blue.$\rbrace$$\mid$$\lbrace$hsb .hue. .saturation. .brightness.$\rbrace$

set colour $\backslash$name$\mid$$\lbrace$rgb .red. .green. .blue.$\rbrace$$\mid$$\lbrace$hsb .hue. .saturation. .brightness.$\rbrace$

set contour format $\backslash$style$\mid$default

set contour label for lines exceeding .x. cm

set contour label position $\lbrace$.start\_cm. .between\_cm.$\rbrace$$\mid$default

set contour labels rotated$\mid$horizontal$\mid$whiteunder$\mid$nowhiteunder

set dash [.n.$\mid$$\lbrace$.dash\_cm. .blank\_cm. ...$\rbrace$$\mid$off]

set environment

set error action to core dump

set flag $\backslash$name [off]

set font color $\backslash$name$\mid$$\lbrace$rgb .red. .green. .blue.$\rbrace$$\mid$$\lbrace$hsb .hue. .saturation. .brightness.$\rbrace$

set font colour $\backslash$name$\mid$$\lbrace$rgb .red. .green. .blue.$\rbrace$$\mid$$\lbrace$hsb .hue. .saturation. .brightness.$\rbrace$

set font size $\lbrace$.size. [cm]$\rbrace$$\mid$default

set font to $\backslash$fontname

set graylevel .brightness.$\mid$white$\mid$black

set greylevel .brightness.$\mid$white$\mid$black

set grid missing $\lbrace$above$\mid$below .intercept. .slope.$\rbrace$$\mid$$\lbrace$inside curve$\rbrace$

set ignore initial newline [off]

set ignore error eof

set image colorscale ...

set image colourscale ...

set image grayscale using histogram [black .bl. white .wh.]

set image greyscale using histogram [black .bl. white .wh.]

set image grayscale [black .bl. white .wh. [increment .inc.]]

set image greyscale [black .bl. white .wh. [increment .inc.]]

set image missing value color to white$\mid$black$\mid$$\lbrace$graylevel .brightness.$\rbrace$

set image missing value colour to white$\mid$black$\mid$$\lbrace$graylevel .brightness.$\rbrace$

set image range .min\_value. .max\_value.

set input data window x$\mid$y $\lbrace$.min. .max.$\rbrace$$\mid$off

set line width [axis$\mid$symbol$\mid$all] .width\_pt.$\mid$$\lbrace$rapidograph $\backslash$name$\rbrace$$\mid$default

set missing value .value.

set postscript filename "$\backslash$string"

set page portrait$\mid$landscape$\mid$$\lbrace$factor .mag.$\rbrace$$\mid$$\lbrace$translate .xcm. .ycm.$\rbrace$

set panel .row. .col.

set panels .rows. .cols. [.dx\_cm. .dy\_cm.]

set symbol size .size\_cm.$\mid$default

set tic size .size.$\mid$default

set tics in$\mid$out

set trace [on$\mid$off]

set u scale .cm\_per\_unit.$\mid$$\lbrace$as x$\rbrace$

set u scale .cm\_per\_unit.

set u scale as x

set v scale .cm\_per\_unit.$\mid$$\lbrace$as y$\rbrace$

set x axis top$\mid$bottom$\mid$increasing$\mid$decreasing$\mid$$\lbrace$.left. .right. [.incBig. [.incSml.]]$\rbrace$$\mid$unknown

set x format $\backslash$format$\mid$off

set x grid .left. .right. .inc.$\mid$$\lbrace$/.cols.$\rbrace$

set x margin $\lbrace$[bigger$\mid$smaller] .size.$\rbrace$ $\mid$ default

set x name "$\backslash$string"

set x size .width\_cm.$\mid$default

set x type linear$\mid$log$\mid$$\lbrace$map E$\mid$W$\mid$N$\mid$S$\rbrace$

set y axis label horizontal$\mid$vertical

set y axis left$\mid$right$\mid$increasing$\mid$decreasing$\mid$$\lbrace$.bottom. .top. [.incBig. [.incSml.]]$\rbrace$$\mid$unknown

set y format $\backslash$format$\mid$off

set y grid .bottom. .top. .inc.$\mid$$\lbrace$/.rows.$\rbrace$

set y margin $\lbrace$[bigger$\mid$smaller] .size.$\rbrace$ $\mid$ default

set y name "$\backslash$string"

set y size .height\_cm.$\mid$default

set y type linear$\mid$log$\mid$$\lbrace$map N$\mid$S$\mid$E$\mid$W$\rbrace$

set z missing above$\mid$below .intercept. .slope.

show all

show axes

show color

show colornames

show columns [statistics]

show flags

show grid [mask]

show hint of the day

show image

show misc

show next line

show traceback

show stopwatch

show synonyms

show time

show variables

show .value. $\mid$ $\lbrace$rpn ...$\rbrace$ $\mid$ "$\backslash$text" [.value.$\mid$$\lbrace$rpn ...$\rbrace$$\mid$text [...]]

smooth $\lbrace$x [.n.]$\rbrace$ $\mid$ $\lbrace$y [.n.]$\rbrace$ $\mid$ $\lbrace$grid data [.f.$\mid$$\lbrace$along x$\mid$y$\rbrace$]$\rbrace$

skip [forward$\mid$backward] [.n.]

sprintf $\backslash$synonym "format" .variable. [.variable. [...]]

state save$\mid$restore$\mid$display

superuser

system $\backslash$system-command

while .test.$\mid$$\lbrace$rpn ...$\rbrace$

write columns to $\backslash$filename

write contour .value. to $\backslash$filename

write grid to $\backslash$filename [bycolumns]

write image colorscale to $\backslash$filename

write image grayscale to $\backslash$filename

write image greyscale to $\backslash$filename

write image mask [pgm$\mid$rasterfile] to $\backslash$filename

write image [pgm$\mid$rasterfile] to $\backslash$filename

?draw axes exploded

?contour xyz data

?set axes

?draw image BW raster


\endexample
% end of INSERT 
\parindent 0ex

\notice

\bye
